\chapter{Conclusion}
\label{chp:conclusion}

\par After around 4 months work of this project, I really gained a lot of knowledge from this project. It is quite hard to begin with this project because it is the project based on previous student's master project and the master thesis of the previous project is not good enough to understand the working process of the system and not clear enough to set up the whole system running. Moreover, there are four different programming languages used in previous project, it makes everything more harder.
\par However, through the research on the internet, especially \gls{rpi} development has a really good developer community, the resources to deal with these problem are plenty for the new beginner like me. Thanks to the Github, the source code repository of previous student work is public to me, then it is really nice for me to try to understand the way other programmer work.
\par The most biggest challenge during the development is the stability of \gls{sd} card file system on \gls{rpi}. The file system on that \gls{sd} card has been ruined by voltage changes twice, every time all the work has been done need to be done again including \gls{rpi} configuration. These experience showed me that it is really important to have a good quality \gls{sd} card when you need to develop something on \gls{rpi}.And to be careful with the all the external equipment you connected with \gls{rpi}.
\par At the end, I am glad to learn five different programming languages(shell script, python, \gls{php}, Java, Objective-c) in this project. It is really fun to code with them.