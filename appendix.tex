\appendix
\addtocontents{toc}{%
  \protect\vspace{1em}% 
  \protect\noindent \bfseries \appendixtocname\protect\par
  \protect\vspace{-.5em}%
 }
 \renewcommand{\chaptername}{\appendixname}
%% include below possible appendices (chapters)
\begin{appendices}
\chapter{Appendix}
\label{chp:appendix}
\section{RPI Residential Access Point Scripts}

\begin{algorithm}[h]
\floatname{algorithm}{Code Snippet}
  \caption{network interface configuration file}
  \label{code:network_interface}
  \begin{verbatim}
  
auto lo

iface lo inet loopback
iface eth0 inet dhcp

allow-hotplug wlan0
iface wlan0 inet static
address 10.0.0.1
netmask 255.255.255.0

post-up /etc/network/if-up.d/iptables_setup.sh
 \end{verbatim}
\end{algorithm}

\begin{algorithm}[h]
\floatname{algorithm}{Code Snippet}
  \caption{getClientPermissions function in configserver.py file}
  \label{code:configserver_py}
  \begin{verbatim}

    def getClientPermissions(self):
            #Start polling thread
            threading.Timer(5.0, self.getClientPermissions).start()
            
            print 'Boot Flag: %s' % self.bootFlag
            
            httpServ = httplib.HTTPConnection("129.241.200.170", 8080)
            httpServ.connect()
            payload = urllib.urlencode(
	            {'macaddress': util.getMacAddress('eth0'),
	             'boot_flag' : self.bootFlag})
            headers = {"Content-type": "application/x-www-form-urlencoded",
            			"Accept": "text/plain"}
            request = httpServ.request('POST',
            			 '/rpi/getclients.php',
            			  payload ,
            			  headers)
            response = httpServ.getresponse()
            print 'Permissions Polling Request Sent'
            
            if response.status == httplib.OK:
                array = response.read()
                data = json.loads(array)
                self.bootFlag = 1;
                
                if len(data) == 0:    
                    print 'No client updates recieved from server'
                
                else:
                    
                    for str in data:
                    
                        client_string = json.dumps(str)
                        print 'Client from server : %s' % client_string
                        client_json = json.loads(client_string)
                        key = client_json ['client']['client_info']['mac']
                        status = 
                        client_json ['client']['permissions']['access']['allow']
	
                        if status == True:
                        	self.clientlist.removeClient(key)
                        	self.clientlist.addClient(client_json)
                    	else:
                    		self.clientlist.blockClient(key)				

                        reload_leases_cmd = 
                        		'bash /etc/reload_dhcp_leases.sh %s' % key 
                        iptablesapi.executeCommand(reload_leases_cmd)
                
            httpServ.close()

 \end{verbatim}
\end{algorithm}

\begin{algorithm}[h]
\floatname{algorithm}{Code Snippet}
  \caption{main functions in clienthandler.py file}
  \label{code:clienthandler_py}
  \begin{verbatim}

    def addClient(self, client_json):      
        ip = self.ipaddress_pool.pop();
        client = Client(client_json, ip)
        
        #  Add static lease to DHCP lease file (mac, ip, lease time)
        add_lease_cmd = 
        	'bash /etc/add_static_lease.sh %s %s 2m' % (client.mac, client.ip)
        
        iptablesapi.executeCommand(add_lease_cmd)
        self.clientlist[client.mac] = client

    def blockClient(self, key):

	self.removeClient(key);
	
	block_mac_cmd = 'bash /etc/block_static_lease.sh %s' % key
	iptablesapi.executeCommand(block_mac_cmd)
		
    
    def removeClient(self, key):
        
        if self.clientlist.has_key(key):
            client = self.clientlist[key]
            ip = client.ip
            self.ipaddress_pool.append(ip)
            
            #Remove static lease from lease file (mac) 
            clear_lease_cmd = 
            'bash /etc/clear_dhcp_lease.sh %s' % client.mac
            
            iptablesapi.executeCommand(clear_lease_cmd)
            #Remove chains and rules from iptables 
            client.deleteInitialChainAndRules()
            del self.clientlist[key]
    	else:
		clear_lease_cmd = 
		'bash /etc/clear_dhcp_lease.sh %s' % key

		iptablesapi.executeCommand(clear_lease_cmd)
    
    def fillIPaddressPool(self):
        #Range for authorized subnet
        for x in xrange(60, 5, -1):
            ip = '10.0.0.%s' % x
            self.ipaddress_pool.append(ip)

 \end{verbatim}
\end{algorithm}

\begin{algorithm}[h]
\floatname{algorithm}{Code Snippet}
  \caption{iptables\_setup.sh}
  \label{code:iptables_setup}
  \begin{verbatim}
  
#DNS server of unathorized subnet
DNS1="129.241.200.170"

#IP range of unthorized subnet
IPunauth="10.0.0.64/27"

#IP range authorized subnet
IPauth="10.0.0.1/26"

#Flush all tables and delete all custom chains
iptables -F
iptables -X
iptables -t nat -F
iptables -t nat -X
iptables -t filter -F
iptables -t filter -X
iptables -t mangle -F
iptables -t mangle -X
iptables -t raw -F
iptables -t raw -X

#Default policies for chains in filter table
iptables -t filter -P INPUT ACCEPT
iptables -t filter -P FORWARD DROP
iptables -t filter -P OUTPUT ACCEPT

#NAT
iptables -A POSTROUTING -t nat -o eth0 -j MASQUERADE

#Allow traffic to and from the management sever
iptables -A FORWARD -s $IPunauth -d $DNS1 -j ACCEPT
iptables -A FORWARD -s $DNS1 -d $IPunauth -j ACCEPT

#Direct http traffic to local proxy to obtain client MAC address
iptables -t nat -A PREROUTING -s $IPunauth 
-p tcp --dport 80 -j REDIRECT --to-ports 8089

#Redirect localy generated DNS traffic
iptables -t nat -A OUTPUT -p udp --dport 53 -j DNAT --to $DNS1
iptables -t nat -A OUTPUT -p tcp --dport 53 -j DNAT --to $DNS1
 \end{verbatim}
\end{algorithm}

\begin{algorithm}[h]
\floatname{algorithm}{Code Snippet}
  \caption{main functions in iptablesapi.py}
  \label{code:iptablesapi}
  \begin{verbatim}
  
#!/usr/bin/python
import subprocess

def createNewClientChain(chain_name):
        new_chain_cmd = 'iptables -N %s' % chain_name
        executeCommand(new_chain_cmd)
        
def deleteClientChain(chain_name):
        delete_chain_cmd = 'iptables -X %s' % chain_name
        executeCommand(delete_chain_cmd)
        
def jumpFromBuiltInChainRule(builtin_chain_name, custom_chain_name, client_ip):
        jump_rule_cmd1 = 
        'iptables -I %s -s %s -j %s'  
        		% (builtin_chain_name, client_ip, custom_chain_name)
        jump_rule_cmd2 = 
        'iptables -I %s -d %s -j %s'  
        		% (builtin_chain_name, client_ip, custom_chain_name)
        executeCommand(jump_rule_cmd1)
        executeCommand(jump_rule_cmd2)

def deleteJumpFromBuiltInChainRule(builtin_chain_name, 
	custom_chain_name, client_ip):
        #delete_jump_rule_cmd = 
        'iptables -D %s -m mac --mac-source %s -j %s'  
        		% (builtin_chain_name, client_mac, custom_chain_name)  
        delete_jump_rule_cmd1 = 
        'iptables -D %s -s %s -j %s'  
        		% (builtin_chain_name, client_ip, custom_chain_name)
        delete_jump_rule_cmd2 = 
        'iptables -D %s -d %s -j %s'  
        		% (builtin_chain_name, client_ip, custom_chain_name)
        executeCommand(delete_jump_rule_cmd1)
        executeCommand(delete_jump_rule_cmd2)

def acceptAllTrafficFromClient(chain_name):
    accept_traffic_cmd = 'iptables -I %s -j ACCEPT' % chain_name
    delete_drop_rule_cmd = 'iptables -D %s -j DROP' % chain_name
    executeCommand(accept_traffic_cmd)
    executeCommand(delete_drop_rule_cmd)

def blockAllTrafficFromClient(chain_name):
    block_traffic_cmd = 'iptables -I %s -j DROP' % chain_name
    delete_accept_rule_cmd = 'iptables -D %s -j ACCEPT' % chain_name
    executeCommand(block_traffic_cmd)
    executeCommand(delete_accept_rule_cmd)

def initialSetup():
    setup_cmd = 'sh /etc/network/if-up.d/iptables_setup.sh'
    executeCommand(setup_cmd)

 \end{verbatim}
\end{algorithm}

\end{appendices}