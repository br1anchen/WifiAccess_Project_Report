\chapter{Central Management Server Improvement}
\label{chp:central_server}

\par There was only one web service on Tomcat 6 in the previous master project to host request web page \ref{fig:wifi_request_page} and provide \gls{api} for mobile management application. The central server in this project has two port hosting two web service for two different use cases.  On the port 8080 is a web service hosting on Tomcat 6 \cite{tomcat} web server to provide the mobile management application \gls{api}. The other web service on port 80 is the default web service on Apache \gls{http} Server \cite{apache} version 2 to provide the simple web page for client request user. The reason to provide the two different web service on different port number is that it can prevent unauthenticated client request user to hack the using mobile application web service to get the authenticated user right for unauthenticated device. Because the time limit for this project, the prototype is to provide two different web service to demonstrate the two web service security mechanism in this case. The source cod e of central server web service is on the Github repository \url{https://github.com/br1anchen/WifiAccess_ManagementServer}.

\section{Technology Using on Remote Central Server}

\subsection{Apache Tomcat}
\par Apache Tomcat (or simply Tomcat, formerly also Jakarta Tomcat) is an open source web server and servlet container developed by the Apache Software Foundation (ASF). Tomcat implements the Java Servlet and the JavaServer Pages (JSP) specifications from Sun Microsystems, and provides a "pure Java" HTTP web server environment for Java code to run in. This project is using Apache Tomcat to provide \gls{api} for mobile management application because it is easy to set up on Linux operating system on remote central server since it has already been installed. Another reason is that the server back-end of remote central server is implemented in \gls{php} 5, it is easy to build on Apache Tomcat and Apache \gls{http} Server for simple web service use and its administrator tool phpmyadmin \cite{phpmyadmin} which is also used in this project is easy to set up as well. The hosting directory on Apache Tomcat is /var/lib/tomcat6/webapp/ROOTl.

\subsection{Apache HTTP Server}
\par The Apache gls{http} Server is a web server application notable for playing a key role in the initial growth of the World Wide Web. Apache is developed and maintained by an open community of developers under the auspices of the Apache Software Foundation. Most commonly used on a Unix-like system(in this project case which is Linux operating system), the software is available for a wide variety of operating systems. The chosen reason is the same for previous section of Apache Tomcat. In this project, it is used to host a simple web page for client request user to submit their basic user information to the administrator and store these information in MySQL\cite{mysql} database. The hosting directory on Apache version 2 \gls{http} Sever is /home/torgier/phpbackend/.

\subsection{PHP: Hypertext Preprocessor}
\par PHP is a server-side scripting language designed for web development but also used as a general-purpose programming language. PHP is now installed on more than 244 million websites and 2.1 million web servers.PHP code is interpreted by a web server with a PHP processor module, which generates the resulting web page: PHP commands can be embedded directly into an HTML source document rather than calling an external file to process data. It has also evolved to include a command-line interface capability and can be used in standalone graphical applications. In the previous master project, the central management server system is implemented in PHP as programming language, so it is still used in this project. The main reason to use PHP is because it is easy to set up and not required highly structured code design for this prototype project. There will be more detail about the function implemented by PHP in later section.

\subsection{MySQL}
\par MySQL is the world's second most widely used open-source relational database management system. MySQL is a popular choice of database for use in web applications, and is a central component of the widely used \gls{lamp} open source web application software stack.For commercial use, several paid editions are available, and offer additional functionality. Applications which use MySQL databases include: TYPO3, MODx, Joomla, WordPress, phpBB, MyBB, Drupal and other software. MySQL is also used in many high-profile, large-scale websites, including Wikipedia, Google (though not for searches), Facebook, Twitter, Flickr, and YouTube. The reason to use in this project system is simply because it is nicely suit for \gls{lamp} open source web application software. MySQL is used as administrator information database and request client information database in this project.

\subsection{phpMyAdmin}
\par phpMyAdmin is a free and open source tool written in PHP intended to handle the administration of MySQL with the use of a web browser. It can perform various tasks such as creating, modifying or deleting databases, tables, fields or rows; executing SQL statements; or managing users and permissions. It is used for testing reason in this project. It is installed on the remote central management server in previous master project, but it is introduced in this report project because it is easy for developer to modify and check the database without running the full implemented system. This application can be running on the web browser, it is good in server and client communication system to debug and test the both client side application and server side web service.

\section{Improvement on Central Management Server}

\subsection{Authentication Log on Service}
\par In the previous master project, the log on function on mobile application is actually only required user name validation since there is no password data stored in the administrator database. Before implementing the more safe authentication mechanism in the system, the password data column is added by phpMyAdmin application from the web browser. Then the new log on \gls{api} shown in Code Snippet \ref{code:loginapp_server} is implemented on the mobile application \gls{api} hosting web service. It added one more \gls{sql} where statement parameter as password, then do the select \gls{sql} query statement to check if there is any administrator data has the same e-mail (which is user name for administrator in this system) and same password. If there is one correct result then the web service will return \gls{http} response with status code number 200, otherwise it will return \gls{http} response with status code number 401 as login failure. The reason to fix user name and password as two input validation to login is that only one input validation login mechanism is too easy to hack and in this case the e-mail address is even more open and public for any one to check and use. It is better to add password as quite normal login input validation in modern login system.

\begin{algorithm}[h]
\floatname{algorithm}{Code Snippet}
  \caption{ /admin/mobile/loginapp.php}
  \label{code:loginapp_server}
  \begin{verbatim}
  
<?php
include_once('../../mysql.php');

$user_email = $_POST['email'];
$user_pwd = $_POST['password'];

$oMySQL = new MySQL('WifiAccess', 'root', 'dork2001');

$where_stmt['email'] = base64_decode($user_email);
$where_stmt['password'] = base64_decode($user_pwd);

$user = $oMySQL->Select('user', $where_stmt);

if(is_null($user['email']) == true){
        HttpResponse::status(401);
}else{
        HttpResponse::status(200);
}

HttpResponse::setContentType('text/html');
HttpResponse::setData('');
HttpResponse::send();

?>
 \end{verbatim}
\end{algorithm}

\subsection{Security Log on HTTP Request}
\par In the previous master project, all the \gls{http} request between client and server is pure string with the authentication information in the request body or header. It is quite vulnerable for the software security view. Any hacker can use the hacking technology like phishing or sniffing to get the correct administrator information to take control of the this internet access control system. So in this project, a simple way to encrypt the administrator login validation information is introduced by this reason. The login function shown in Code Snippet \ref{code:loginapp_server}, it is implemented with Base64\cite{base64} decoding just for demonstration of the encryption process on central management server web service. Base64 encoding schemes are commonly used when there is a need to encode binary data that needs to be stored and transferred over media that is designed to deal with textual data. This is to ensure that the data remains intact without modification during transport. Base64 is commonly used in a number of applications including email via \gls{mime}, and storing complex data in \gls{xml}. Since the login \gls{http} post request is with Base64 encoding data, then the encoding process in the mobile application need to be implemented as well, it will be discussed in the Chapter \ref{chp:mobile_app}.

\subsection{E-mail Notification Mechanism}
\par In the previous master project, it is using Google Cloud Message\cite{gcm} for Android application to implement the push notification function. \gls{gcm} is a service that allows you to send data from your server to your users' Android-powered device, and also to receive messages from devices on the same connection. The \gls{gcm} service handles all aspects of queueing of messages and delivery to the target Android application running on the target device. \gls{gcm} is completely free no matter how big your messaging needs are, and there are no quotas. It works well for Android Application, but in this project the \gls{ios} application is the main application need to be implemented in the system, then \gls{gcm} is not the case for Apple Push notification service \cite{apns}. Apple Push Notification service is the centerpiece of the push notifications feature. It is a robust and highly efficient service for propagating information to iOS and OS X devices. Each device establishes an accredited and encrypted IP connection with the service and receives notifications over this persistent connection. If a notification for an application arrives when that application is not running, the device alerts the user that the application has data waiting for it. The Apple notification service is required for Apple Developer Account to use, since in this project, the developer is using some company \gls{ios} developer account, then it can not be used for further using than prototype developing. Afterwards, the E-mail notification mechanism is introduced by this reason. The mechanism is implemented based on third party \gls{php} class PHPMailer, which is included in the Github repository as well. The implemented code is shown in Code Snippet \ref{code:email_server}. The notification E-mail is sent by the Google mail service from the sender E-mail address "wifi.access.manager@gmail.com". Once the user submit the internet access request from the connecting device then the web service will send an E-mail with the request user name in the mail body to the responsible administrator which registered in the administrator information database on the server according to responsible \gls{rpi} residential access point. In this mechanism, the administrator can get up to date the request notification by E-mail but only the request is his responsibility.

\begin{algorithm}[h]
\floatname{algorithm}{Code Snippet}
  \caption{ e-mail notification function in requestaccess.php}
  \label{code:email_server}
  \begin{verbatim}
  
require_once('classes/class.phpmailer.php');

$phpmailer = new PHPMailer();

if(!empty($client_mac) && !empty($rpi_mac)){
        
        $oMySQL = new MySQL('WifiAccess', 'root', 'dork2001');
        $query = "INSERT INTO client (mac, name, rpi_mac, email) 
        		VALUES ('$client_mac','$name','$rpi_mac','$email') 
        		ON DUPLICATE KEY UPDATE mac = VALUES(mac),
        			 name = VALUES(name), 
        			 rpi_mac = VALUES(rpi_mac), 
        			 email = VALUES(email)";
        $oMySQL->ExecuteSQL($query);
        
        $response = sendPushNotification($rpi_mac, $client_mac, $name);

        $mailSQL = new MySQL('WifiAccess','root','dork2001');
        $where['rpi_mac'] = $rpi_mac;
        $user = $mailSQL->Select('user', $where);
        if(!is_null($user['email'])){
                $phpmailer->IsSMTP();
                $phpmailer->Host       = "ssl://smtp.gmail.com";
                $phpmailer->SMTPAuth   = true;
                $phpmailer->Port       = 465;
                $phpmailer->Username   = "wifi.access.manager@gmail.com";
                $phpmailer->Password   = "ntnu2013";

                $phpmailer->SetFrom('wifi.access.manager@gmail.com', 'Wifi Manager');

                $phpmailer->Subject    = "New Request User";
                $phpmailer->Body 
                		= "There is a new user($name) request for wifi access!";

                $phpmailer->AddAddress($user['email']);

                if(!$phpmailer->Send()) {
                        echo "Mailer Error: " . $phpmailer->ErrorInfo;
                } else {
                        echo "Message sent!";
                }
        }

}
 \end{verbatim}
\end{algorithm}